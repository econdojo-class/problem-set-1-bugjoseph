%------------%
%  Preamble  %
%------------%

\documentclass[final,11pt]{article}
\usepackage[paperwidth=9.0in, top=1.2in, bottom=1.2in, left=1.2in, right=1.2in]{geometry}
\usepackage{amsmath}
\usepackage{color}
\usepackage{multirow}
\usepackage{setspace}
\usepackage{fancyhdr}
\usepackage{longtable}
\usepackage{array}
\usepackage{booktabs}
\usepackage{mathpazo}
\usepackage{threeparttable}
\usepackage{eurosym}
\usepackage[colorlinks, linkcolor=blue, anchorcolor=blue, citecolor=blue]{hyperref}

\renewcommand{\headrulewidth}{0pt}
\setlength{\arraycolsep}{10pt}
\setlength\headheight{0.5cm}
\setlength\headsep{0.8cm}
\setlength\footskip{1.0cm}
\setlength{\parindent}{0em}
\pagestyle{fancy}
\chead{\textcolor[rgb]{0.5,0.5,0.5}{\sc Spring 2025: ECON 3120}}

%------------%
%  Document  %
%------------%

\begin{document}
\thispagestyle{empty}
\begin{spacing}{1.25}

\textbf{Your Name:\hfill Problem Set 1 Due: Feb. 18, 2025}\\

(1) The Poisson distribution has probability mass function
\begin{gather}
    p(y_i|\theta)=\frac{\theta^{y_i}e^{-\theta}}{y_i!},\qquad \theta>0,\qquad y_i=0,1,\ldots
\end{gather}
and let $y_1,\ldots,y_n$ be random sample from this distribution.
\begin{enumerate}
    \item Show that the gamma distribution $\mathcal{G}(\alpha,\beta)$ is a conjugate prior distribution for the Poisson distribution.
    \item Show that $\bar{y}$ is the MLE for $\theta$.
    \item Write the mean of the posterior distribution as a weighted average of the mean of the prior distribution and the MLE.
    \item What happens to the weight on the prior mean as $n$ becomes large?
\end{enumerate}

\begin{enumerate}
    \item The Gamma prior is given by 
    \begin{gather}
    \pi(\theta|\alpha, \beta)\propto \theta^{\alpha-1}e^{-\frac{\theta}{\beta}}.
    \end{gather}
\end{enumerate}
    \begin{enumerate}
        \item The likelihood function is given by
        \begin{gather}
        f(y|\theta)\propto\theta^{y_i}e^{}
            
        \end{gather}
    \end{enumerate}
(2) Consider the following two sets of data obtained after tossing a die 100 and 1000 times, respectively:
\begin{center}
    \begin{tabular}{ r r r r r r r }
        \hline
        $n$ & 1 & 2 & 3 & 4 & 5 & 6 \\
        \hline
        100 & 19 & 12 & 17 & 18 & 20 & 14 \\  
        1000 & 190 & 120 & 170 & 180 & 200 & 140 \\
        \hline
    \end{tabular}
\end{center}
Suppose you are interested in $\theta_1$, the probability of obtaining a one spot. Assume your prior for all the probabilities is a Dirichlet distribution, where each $\alpha_i=2$. Compute the posterior distribution for $\theta_1$ for each of the sample sizes in the table. Plot the resulting distribution and compare the results. Comment on the effect of having a larger sample.

\end{spacing}
\end{document}